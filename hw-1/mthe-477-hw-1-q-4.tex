% region Filename parsing.
% Provides macros manipulating strings of tokens.
\RequirePackage{xstring}

% Store the jobname as a string with category 11 characters.
\edef\normaljobname{\expandafter\scantokens\expandafter{\jobname\noexpand}}
\StrBetween{\normaljobname}{hw-}{-q}[\homeworknumber]
\StrBehind{\normaljobname}{-q-}[\questionnumber]
% endregion

\documentclass[
  coursecode={MTHE 477},
  assignmentname={Homework \homeworknumber},
  studentnumber=20053722,
  name={Bryan Hoang},
  draft,
  % final,
]{
  ltxanswer%
}

\usepackage{bch-style}

\begin{document}
  \begin{questions}
    \setcounter{question}{\questionnumber}
    \addtocounter{question}{-1}
    \question[25]{}
    \begin{solution}
      From the textbook, the capacity of the \(q\)-ary symmetric DMC is
      \begin{equation*}
        C = \log_{2}q + \varepsilon\log_{2}\frac{\varepsilon}{q-1} + (1-\varepsilon)\log_{2}(1-\varepsilon).
      \end{equation*}
      With \(\abs{\mathcal{Z}}=q\), the rate distortion function of the \(q\)-ary symmetric DMS is
      \begin{equation*}
        R(D) = \begin{cases}
          \log_{2}q - D\log_{2}(q-1) - H_{b}(D), &\text{if } D \in [0, \frac{q-1}{q}], \\
          0,                                     &\text{if } D \ge \frac{q-1}{q}.
        \end{cases}
      \end{equation*}
      Then we want to see if the uncoded source-channel transmission scheme can achieve the Shannon limit of the communication system for it to be optiml. That is,
      \begin{align*}
        R(D_{SL})                                                &= C                                                                                                        \\
        \cancel{\log_{2}q} - D_{SL}\log_{2}(q-1) - H_{b}(D_{SL}) &= \cancel{\log_{2}q} + \varepsilon\log_{2}\frac{\varepsilon}{q-1} + (1-\varepsilon)\log_{2}(1-\varepsilon) \\
        D_{SL}\log_{2}(q-1) + H_{b}(D_{SL})                      &= \varepsilon\log_{2}(q-1) + H_{b}(\varepsilon)
        \intertext{\because\ \(x\log_{2}(q-1) + H_{b}(x)\) is injective on \(x\in[0, \frac{q-1}{q}]\),}
        D_{SL}                                                   &= \varepsilon.\numberthis\label{eq:shannon-limit}
      \end{align*}
      With that fact, let's see if the distortion constraint is satisfied for the scheme with \(R_{sc}=1\) source symbol/channel use. First, let \(W\) denote the additive noise of the channel (mod \(q\)). Then the expected distortion is
      \begin{align*}
        Ed(Z,\hat{Z}) &= P(Z \ne \hat{Z})                                         \\
                      &= P(W \ne 0)                                               \\
                      &= 1-(1-\varepsilon)                                        \\
                      &= \varepsilon                                              \\
                      &= D_{SL}            & &\text{by~\eqref{eq:shannon-limit}.}
      \end{align*}
      Thus, the proposed scheme achieves the Shannon limit of the communication system, and thus \fbox{is optimal} for the communication system.
    \end{solution}
  \end{questions}
\end{document}
