% region Filename parsing.
% Provides macros manipulating strings of tokens.
\RequirePackage{xstring}

% Store the jobname as a string with category 11 characters.
\edef\normaljobname{\expandafter\scantokens\expandafter{\jobname\noexpand}}
\StrBetween{\normaljobname}{hw-}{-q}[\homeworknumber]
\StrBehind{\normaljobname}{-q-}[\questionnumber]
% endregion Filename parsing.

\documentclass[
  coursecode={MTHE 477},
  assignmentname={Homework \homeworknumber},
  studentnumber=20053722,
  name={Bryan Hoang},
  draft,
  % final,
]{
  ltxanswer%
}

\usepackage{bch-style}

\begin{document}
  \begin{questions}
    \setcounter{question}{\questionnumber}
    \addtocounter{question}{-1}
    \question[25]\
    \begin{parts}
      \part{}
      \begin{solution}
        For the first-order linear predictor of Coder 1, let \(a\) be the prediction coefficient to be determined. Then the orthogonality principle states that
        \begin{align*}
          \E{(X_{n} - a X_{n - 1}) X_{n - 1}}       &= 0                                                                                                         \\
          \E{X_{n} X_{n - 1}} - a \E{X_{n - 1}^{2}} &= 0                                                                                                         \\
          a                                         &= \frac{\E{X_{n} X_{n - 1}}}{\E{X_{n - 1}^{2}}}                                                             \\
                                                    &= \frac{r_{1}}{r_{0}}                           & &\text{\(\because\ \{X_{n}\}\)  is wide sense stationary} \\
          \alignedbox{                              &= \frac{1}{2}}.
        \end{align*}
        For the second-order linear predictor of Coder 2, let \(a_{1}\) and \(a_{2}\) be the prediction coefficients to be determined. Then the orthogonality principle states that
        \begin{equation*}
          \E{(X_{n} - a_{1} X_{n - 1} - a_{2} X_{n - 2}) X_{n - i}} = 0 \qquad \text{for \(i = 1, 2\)}
        \end{equation*}
        which gives the following system of linear equations:
        \begin{align*}
                                   &\begin{cases}
                                      a_{1} r_{0} + a_{2} r_{1} = r_{1}, \\
                                      a_{1} r_{1} + a_{2} r_{0} = r_{2}
                                    \end{cases} \\
          \Rightarrow              &\begin{cases}
                                      a_{1} \cdot 4 + a_{2} \cdot 2 = 2, \\
                                      a_{1} \cdot 2 + a_{2} \cdot 4 = 1
                                    \end{cases} \\
          \Rightarrow \alignedbox{ &\begin{cases}
                                        a_{1} = \frac{1}{2}, \\
                                        a_{2} = 0
                                      \end{cases}}.
        \end{align*}
      \end{solution}

      \part{}
      \begin{solution}
        Let \(\Delta_{i}\) denote the step size, \(N_{i}\) denote the quantization level, and \(Q_{i}\) denote the uniform quantizer of Coder \(i\). Then since \(\delta_{i} = \frac{2B}{N_{i}}\), we get
        \begin{align*}
          \E*{(e_{n} - Q_{i}(e_{n}))^{2}} &\approx \frac{\Delta_{i}^{2}}{12}              & &\text{\(\because\) the overload distortion is negligible and the prediction error is uniform} \\
                                          &= \frac{\bigl(\frac{2 B}{N_{i}}\bigr)^{2}}{12}                                                                                                  \\
                                          &= \frac{B^{2}}{3 N_{i}^{2}}.                                                                                                                    \\
        \end{align*}
        Then overall system gain for Coder \(i\) (System SNR, not in dB) is given by
        \begin{align*}
          G^{i} &= G_{clp}^{i} G_{Q_{i}}                                                                      \\
                &\approx G_{olp}^{i} G_{Q_{i}}                                                                \\
                &\approx \frac{r_{0}}{r_{0} - r_{1} a_{1}} \cdot \frac{\sigma^{2}}{\frac{B^{2}}{3 N_{i}^{2}}} \\
                &= \frac{r_{0}}{r_{0} - r_{1} a_{1}} \cdot \frac{3 N_{i}^{2} \sigma^{2}}{B^{2}}.
        \end{align*}
        Therefore, the approximate SNR improvement is
        \begin{align*}
          10 \log_{10} \frac{G^{2}}{G^{1}} &= 10 \log_{10} \frac{N_{2}^{2}}{N_{1}^{2}} \\
                                           &= 10 \log_{10} \frac{256^{2}}{64^{2}}      \\
          \alignedbox{                     &\approx \qty{12}{\decibel}}.
        \end{align*}
      \end{solution}
    \end{parts}
  \end{questions}
\end{document}
