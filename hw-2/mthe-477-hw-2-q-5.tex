% region Filename parsing.
% Provides macros manipulating strings of tokens.
\RequirePackage{xstring}

% Store the jobname as a string with category 11 characters.
\edef\normaljobname{\expandafter\scantokens\expandafter{\jobname\noexpand}}
\StrBetween{\normaljobname}{hw-}{-q}[\homeworknumber]
\StrBehind{\normaljobname}{-q-}[\questionnumber]
% endregion

\documentclass[
  coursecode={MTHE 477},
  assignmentname={Homework \homeworknumber},
  studentnumber=20053722,
  name={Bryan Hoang},
  draft,
  % final,
]{
  ltxanswer%
}

\usepackage{bch-style}

\begin{document}
  \begin{questions}
    \setcounter{question}{\questionnumber}
    \addtocounter{question}{-1}
    \question[20]\
    \begin{parts}
      \part{}
      \begin{solution}
        The LZ78 parsing of the sequence \(x^{n}=11111111\ldots1\) is
        \begin{equation*}
          \boxed{1,11,111,1111,\dotsc,\underbrace{1\ldots1}_{l_{n}}},
        \end{equation*}
        where the length of the last phrase, \(l_{n}\), is
        \begin{equation*}
          l_{n} = \begin{cases}
            T_{n}                 &\text{if \(n\) is a triangular number,} \\
            \lfloor T_{n} \rfloor &\text{otherwise,}
          \end{cases}
        \end{equation*}
        and where
        \begin{equation*}
          T_{n} = \frac{\sqrt{8n + 1} - 1}{2},
        \end{equation*}
        is the triangular root of \(n\).
      \end{solution}

      \part{}
      \begin{solution}
        \begin{proof}
          We know that
          \begin{equation*}
            l(x^{n}) = c(x^{n}) (\log c(x^{n}) + O(1)),
          \end{equation*}
          where \(c(x^{n})\) denotes the number of phrases in the dictionary obtained by parsing \(x^{n}\). In this case, \(c(x^{n})=T_{n}\) from part~(\ref{part@5@1}). We also know that
          \begin{equation}\label{eq:triangular-number-upper-bound}
            T_{n} = \frac{\sqrt{8n + 1} - 1}{2} \le \sqrt{2n}.
          \end{equation}
          Therefore,
          \begin{align*}
            \lim_{n\to\infty} \frac{1}{n} l(x^{n}) &= \lim_{n\to\infty} \frac{1}{n} T_{n} (\log T_{n} + O(1))                                                                   \\
                                                   &\le \lim_{n\to\infty} \frac{1}{n} \sqrt{2n} (\log \sqrt{2n} + O(1))   & &\text{by~\eqref{eq:triangular-number-upper-bound}} \\
                                                   &= \lim_{n\to\infty} \frac{\sqrt{2} (\log \sqrt{2n} + O(1))}{\sqrt{n}}                                                       \\
                                                   &= 0.
          \end{align*}
        \end{proof}
      \end{solution}
    \end{parts}
  \end{questions}
\end{document}
