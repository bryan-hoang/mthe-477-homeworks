% region Filename parsing.
% Provides macros manipulating strings of tokens.
\RequirePackage{xstring}

% Store the jobname as a string with category 11 characters.
\edef\normaljobname{\expandafter\scantokens\expandafter{\jobname\noexpand}}
\StrBetween{\normaljobname}{hw-}{-q}[\homeworknumber]
\StrBehind{\normaljobname}{-q-}[\questionnumber]
% endregion

\documentclass[
  coursecode={MTHE 477},
  assignmentname={Homework \homeworknumber},
  studentnumber=20053722,
  name={Bryan Hoang},
  draft,
  % final,
]{
  ltxanswer%
}

\usepackage{bch-style}

\begin{document}
  \begin{questions}
    \setcounter{question}{\questionnumber}
    \addtocounter{question}{-1}
    \question[20]\
    \begin{parts}
      \part{}
      \begin{solution}
        \begin{proof}
          We have
          \begin{gather*}
            2^{-l(j)} \le p(j) < 2^{-l(j) + 1} \\
            \Rightarrow -l(j) \le \log_{2} p(j) < -l(j) + 1 \\
            \Rightarrow l(j) \ge -\log_{2} p(j) > l(j) - 1.
          \end{gather*}
          Thus, \(l(j)\) is the lowest integer that is greater than or equal to \(-\log_{2} p(j)\). Therefore, we can conclude that \(l(j) = \lceil -\log_{2} p(j) \rceil\), so the expected code length satisfies \(L(C) \le H(X) + 1\).
        \end{proof}
      \end{solution}

      \part{}
      \begin{solution}
        \begin{proof}
          Let's prove the result by contradiction. First, suppose that \(C\) is not a prefix code. That is, let \(i,j\in\X:i<j\) and assume that \(C(i)\) is a prefix of \(C(j)\), i.e., \(C(i)\) appears in the first \(l(i)\) bits of \(C(j)\). Then \(\hat{F}(i)\) and \(\hat{F}(j)\) also share the first \(l(i)\) bits. Thus, we have
          \begin{equation}\label{eq:lt}
            \hat{F}(j) - \hat{F}(i) < 2^{-l(i)}.
          \end{equation}
          But we also have
          \begin{align*}
            \hat{F}(j) - \hat{F}(i) &= \sum_{k=i}^{j-1} p(k) & &\text{by the definition of \(\hat{F}(j)\)}                    \\
                                    &\ge p(i)                                                                                 \\
                                    &\ge 2^{-l(i)}           & &\text{by the definition of \(l(i)\)}\numberthis\label{eq:ge}.
          \end{align*}
          With~\eqref{eq:lt} and~\eqref{eq:ge}, we have a contradiction. Therefore, \(C\) is a prefix code.
        \end{proof}
      \end{solution}
    \end{parts}
  \end{questions}
\end{document}
