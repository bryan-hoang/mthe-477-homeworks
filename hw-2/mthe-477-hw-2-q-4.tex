% region Filename parsing.
% Provides macros manipulating strings of tokens.
\RequirePackage{xstring}

% Store the jobname as a string with category 11 characters.
\edef\normaljobname{\expandafter\scantokens\expandafter{\jobname\noexpand}}
\StrBetween{\normaljobname}{hw-}{-q}[\homeworknumber]
\StrBehind{\normaljobname}{-q-}[\questionnumber]
% endregion

\documentclass[
  coursecode={MTHE 477},
  assignmentname={Homework \homeworknumber},
  studentnumber=20053722,
  name={Bryan Hoang},
  draft,
  % final,
]{
  ltxanswer%
}

\usepackage{bch-style}

\begin{document}
  \begin{questions}
    \setcounter{question}{\questionnumber}
    \addtocounter{question}{-1}
    \question[20]{}
    \begin{solution}
      \begin{proof}
        Suppose that \(C_{n}\) is the Shannon-Fano code for \(p(x^{n})\). Then for \(i = 1, \dotsc, M\),
        \begin{align*}
          \lim_{n\to\infty} \frac{1}{n} D(p_{i}||p) &= \lim_{n\to\infty} \frac{1}{n} \sum_{x^{n}\in\X^{n}} p_{i}(x^{n}) \log \frac{p_{i}(x^{n})}{p(x^{n})}                                                                                         \\
                                                    &= \lim_{n\to\infty} \frac{1}{n} \sum_{x^{n}\in\X^{n}} p_{i}(x^{n}) \log \frac{p_{i}(x^{n})}{\sum_{j=1}^{M} \alpha_{j} p_{j}(x^{n})}                                                           \\
                                                    &\le \lim_{n\to\infty} \frac{1}{n} \sum_{x^{n}\in\X^{n}} p_{i}(x^{n}) \log \frac{p_{i}(x^{n})}{\alpha_{i} p_{i}(x^{n})}                                                                        \\
                                                    &= \lim_{n\to\infty} \frac{1}{n} (-\log \alpha_{i})                                                                                                                                            \\
                                                    &= 0                                                                                                                                 & &\because \alpha_{i} > 0\quad \forall i = 1, \dotsc, M.
        \end{align*}
        Thus, the code sequence \(\{C_{n}\}\) is universal with respect to \(\mathcal{P}\).
      \end{proof}
    \end{solution}
  \end{questions}
\end{document}
