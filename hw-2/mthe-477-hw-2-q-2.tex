% region Filename parsing.
% Provides macros manipulating strings of tokens.
\RequirePackage{xstring}

% Store the jobname as a string with category 11 characters.
\edef\normaljobname{\expandafter\scantokens\expandafter{\jobname\noexpand}}
\StrBetween{\normaljobname}{hw-}{-q}[\homeworknumber]
\StrBehind{\normaljobname}{-q-}[\questionnumber]
% endregion

\documentclass[
  coursecode={MTHE 477},
  assignmentname={Homework \homeworknumber},
  studentnumber=20053722,
  name={Bryan Hoang},
  draft,
  final,
]{
  ltxanswer%
}

\usepackage{bch-style}

\begin{document}
  \begin{questions}
    \setcounter{question}{\questionnumber}
    \addtocounter{question}{-1}
    \question[20]\
    \begin{parts}
      \part{}
      \begin{solution}
        With the given transition matrix, we can determine the stationary distribution of the Markov source to be \(\pi = (\frac{1}{2}, \frac{1}{2})\). To calculate \(\bar{T}(x^{n})\)
        \begin{equation*}
          \bar{T}(x^{n}) = \hat{F}(x^{n}) + \frac{1}{2} p(x^{n}),
        \end{equation*}
        we will first calculate \(\hat{F}(x^{n})\) using
        \begin{equation*}
          \hat{F}(x^{k}) = \hat{F}(x^{k-1}) + p(x^{k-1}) \sum_{y^{k} < x^{k}} p(y_{k}|x^{k-1}),
        \end{equation*}
        for \(k\in\Z_{\ge2}\). Thus,
        \begin{align*}
          \hat{F}(1)      &= p(1) = \pi(1) = \frac{1}{2}                                                                                                                             \\
          \hat{F}(10)     &= \hat{F}(1) = \frac{1}{2}                                                                                                                                \\
          \hat{F}(100)    &= \hat{F}(10) = \frac{1}{2}                                                                                                                               \\
          \hat{F}(1001)   &= \hat{F}(100) + p(100)p(0|100) = \frac{1}{2} + \frac{1}{2} \cdot \frac{2}{3} \cdot \frac{1}{3} \cdot \frac{1}{3} = \frac{29}{54}                         \\
          \hat{F}(10011)  &= \hat{F}(1001) + p(1001)p(0|1001) = \frac{29}{54} + \frac{1}{2} \cdot \frac{2}{3} \cdot \frac{1}{3} \cdot \frac{2}{3} \cdot \frac{2}{3} = \frac{95}{162} \\
          \hat{F}(100110) &= \hat{F}(10011) = \frac{95}{162}
        \end{align*}
        Therefore,
        \begin{equation*}
          \bar{T}(100110) = \frac{95}{162} + \frac{1}{2}p(100110) = \boxed{\frac{289}{486} = 0.5946502058}.
        \end{equation*}
      \end{solution}

      \part{}
      \begin{solution}
        Given the codeword 00001110000, we compute
        \begin{align*}
          z = \sum_{i=1}^{11} b_{i} 2^{-i} = 2^{-5} + 2^{-6} + 2^{-7} = \frac{7}{128} = 0.0546875.
        \end{align*}
        Then starting from \(k = 1\), we find sequentially
        \begin{equation*}
          \max_{y \in \X : \hat{F}(x^{k-1}y) < z} y
        \end{equation*}
        and let \(x_{k} = y\). We then repeat this for \(k = 2, \dotsc, n\). The resulting \(x^{n}\) will satisfy \(\hat{F}(x^{n}) < z < F(x^{n})\), so we decode \(x^{n}\).

        Note that \(p(a) + p(b) = 0.2 + 0.3 = 0.5\).

        For \(k = 1\):
        \begin{gather*}
          \hat{F}(a) = 0 < z\\
          \hat{F}(b) = 0.2 > z \\
          \Rightarrow x_{1} = a.
        \end{gather*}
        For \(k = 2\):
        \begin{gather*}
          \hat{F}(aa) = \hat{F}(a) = 0 < z\\
          \hat{F}(ab) = \hat{F}(a) + p(a)p(a) = 0 + 0.2 \cdot 0.2 = 0.04 < z \\
          \hat{F}(ac) = \hat{F}(a) + p(a)(p(a) + p(b)) = 0 + 0.2 (0.2 + 0.3)  = 0.1 > z \\
          \Rightarrow x_{2} = b.
        \end{gather*}
        For \(k = 3\):
        \begin{gather*}
          p(ab) = 0.2 \cdot 0.3 = 0.06 \\
          \hat{F}(aba) = \hat{F}(ab) = 0.04 < z \\
          \hat{F}(abb) = \hat{F}(ab) + p(ab)p(a) = 0.04 + 0.06 \cdot 0.2 = 0.052 < z \\
          \hat{F}(abc) = \hat{F}(ab) + p(ab)(p(a) + p(b)) = 0.04 + 0.06 \cdot 0.5 = 0.07 > z \\
          \Rightarrow x_{3} = b.
        \end{gather*}
        For \(k = 4\):
        \begin{gather*}
          p(abb) = 0.2 \cdot 0.3 \cdot 0.3 = 0.018 \\
          \hat{F}(abba) = \hat{F}(abb) = 0.052 < z\\
          \hat{F}(abbb) = \hat{F}(abb) + p(abb)p(a) = 0.052 + 0.018 \cdot 0.2 = 0.0556 > z \\
          \Rightarrow x_{4} = a.
        \end{gather*}
        For \(k = 5\):
        \begin{gather*}
          p(abba) = 0.2 \cdot 0.3 \cdot 0.3 \cdot 0.2 = 0.0036 \\
          \hat{F}(abbaa) = \hat{F}(abba) = 0.052 < z\\
          \hat{F}(abbab) = \hat{F}(abba) + p(abba)p(a) = 0.052 + 0.0036 \cdot 0.2 = 0.05272 < z \\
          \hat{F}(abbac) = \hat{F}(abba) + p(abba)(p(a) + p(b)) = 0.052 + 0.0036 \cdot 0.5 = 0.0538 < z \\
          \Rightarrow x_{5} = c.
        \end{gather*}
        \therefore\ \(\boxed{x^{5} = x_{1}x_{2}x_{3}x_{4}x_{5} = abbac}\).
      \end{solution}

      \part{}
      \begin{solution}
      \end{solution}
    \end{parts}
  \end{questions}
\end{document}
